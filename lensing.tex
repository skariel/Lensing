\documentclass{article}
\usepackage{graphicx, amssymb}

\begin{document}

\title{Gravitational Lensing Expressions}

\maketitle

\section{---}

In the following I use O, S and L to denote the Observer, Source and Lens positions, respectively. P denotes any point in the ray of light we deal with. The lensing equation is
\begin{equation}
    \label{eq1}
    \frac{d^2u}{d\theta^2}+u=\frac{3}{2}r_gu^2
\end{equation}
where $u\equiv1/r$, the radius $r\equiv LP$, and $\theta\equiv \measuredangle SLP$. The gravitational radius $r_g$ is the Schwarzschild radius 
\begin{equation}
    \label{eq2}
    r_g=\frac{2MG}{c^2}
\end{equation}
where $c$ is the speed of light and M the lens mass.

For thin lens approximation I use a couple of definitions: the "breaking point" (B) is a point on the incomming ray at which $\measuredangle SBL=\pi/2$, the incoming ray angle $\phi\equiv \measuredangle BSL$ and the impact parameter $b\equiv BL$. Given some $\phi$, the breaking point B and the impact parameter $b$ are calculated, the incoming ray is deflected at B (stays the same before B) such that $\phi_{afterB}=\phi-2r_g/b$. This assumes that the impact parameter $b$ is much larger than the gravitational radius $r_g$ (this is the thins lens approximation assumption)

Calculating the time of travel is done as follows:
\begin{equation}
    \label{eq3}
    \Delta t_{ray} = \int_{Ray} \frac{dp}{c} \left( 1- \frac{r_g}{r(p)}\right)^{-\frac{1}{2}} 
\end{equation}
the square root term is the regular gravitational time dilation.

For extended masses I think changing $r_g$ to depend on the inner mass at P, the numerics should work just fine for both thin and thick lenses. The metric outside a static spherical symmetric mass doesn't care of the exact inner structure.

Thoughts, comments?  


\end{document}